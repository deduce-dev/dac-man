\section{Staging}
For every dataset, \systemname saves all metadata and index
information in a staging area. Each directory in
the staging area uniquely identifies the dataset (using a hash
representation of the dataset path) indexed by \systemname.
By default, this staging area for is located in
\texttt{\$HOME/.dacman/data}. However, the staging
area can be changed to a custom location through the
command-line. You can change the staging area by using
the following command:

\texttt{\$ \appcmd index mydir/ -s mystage}

The command above creates the indexes inside \texttt{mystage}
directory. You can copy or move these indexes to compare
and calculate the changes, without necessarily copying or
moving the data. This is specifically useful when the datasets
to be compared are located on different systems. The example
below shows how can the staged indexes and metadata information
be copied and compared for finding changes, without copying
the data itself.

\texttt{\$ scp -r user:pwd@~/.dacman/data/ /path/to/mystage/}\\
\texttt{\$ \appcmd diff /path/to/localdir/ /remotedir/ -s /path/to/mystage}

\section{Plug-ins}
By default, \systemname compares the data by transforming file data
into \systemname records, that uses one-dimensional arrays as their
underlying data structure.

You can also use plug-ins by providing external scripts. You can use
your own custom scripts as plug-ins by simply providing the path to
the script. For example, \texttt{myscript} can be used as a plug-in:

\texttt{\$ \appcmd diff /old/path/file1 /new/path/file1 -p myscript}

The command above uses \texttt{myscript} as an external plug-in to compare
the contents of files \texttt{/old/path/file1} and \texttt{/new/path/file1}
instead of the default \systemname data comparator. If you want to
use Unix diff to compare all the modified files in the directories
\texttt{dir1} and {dir2}, run the following command:

\texttt{\$ \appcmd diff /path/to/dir1 /path/to/dir2 --detailed -p /usr/bin/diff}

The \texttt{--detailed} option tells \systemname to compare the data
within the files of the two directories.

Finally, you can build your own plug-ins by extending the ComparatorBase
class. Please refer to the document PluginBuilderGuide.pdf for details. 

\section{Logging}
\systemname uses the standard Python logging for creating execution
logs. The default logging configuration is saved in \texttt{\$HOME/.dacman/config/logging.yaml}
file. \systemname logs all INFO level messages, and prints
messages with levels equal to or over the WARNING level. However,
you can configure the logging as per your requirement by modifying the
configuration file. 
