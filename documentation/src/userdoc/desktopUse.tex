
\section{Quick Tutorial}
To capture changes between two directories \texttt{dir1}
and \texttt{dir2}, run the following command using the \systemname
command-line:

\texttt{\$ \appcmd diff dir1 dir2}

The above command identifies the number of files changed
between the two directories. In order to retrieve detailed
infromation about the changes, you can use the following
command:

\texttt{\$ \appcmd diff dir1 dir2 --detailed}

\section{Command-line}
\systemname enables change capture and analysis in four
simple steps, which provide flexibility to the users in
identifying and capturing changes. \systemname provides
four command-line options to manage each of these steps
separately.

\subsection{scan}
This option scans and saves the directory structure and
other metadata related to a data path. You can specify
an optional staging directory, where the metadata information
will be saved.

\texttt{\$ \appcmd scan <path> [-s STAGINGDIR] [-i [IGNORE [IGNORE ...]]] [--nonrecursive] [--symlinks]}

The arguments to the command are:

\fbox{
  \parbox{\textwidth}{    
$-$s STAGINGDIR            : directory where filesystem metadata and indexes are saved. \\
$-$i [IGNORE [IGNORE ...]] : list of file types to be ignored.\\
$--$nonrecursive           : do not scan the directory contents recursively.\\
$--$symlinks               : include symbolic links.
  }
}

\subsection{index}
This command indexes the files, mapping the files to their contents. 

\texttt{\$ \appcmd index <path> [-s STAGINGDIR] [-m {python,tigres,mpi}]}

The arguments to the command are:

\fbox{
  \parbox{\textwidth}{    
$-$s STAGINGDIR          : directory where filesystem metadata and indexes are saved. \\
$-$m {python,tigres,mpi} : index manager for parallelizing the index creation.
                         The options are python/mpi/tigres. By default, it uses the Python
                         multiprocessing module (manager=python) that is suitable for parallelizing
                         on one node. For multi-node parallelism, users can select between
                         MPI (manager=mpi) or tigres (manager=tigres).
  }
}

\subsection{compare}
This command compares two datapaths. It compares and calculates
the different types of changes.

\texttt{\$ \appcmd compare <oldpath> <newpath> [-s STAGINGDIR]}

The arguments to the command are:

\fbox{
  \parbox{\textwidth}{    
$-$s STAGINGDIR          : directory where filesystem metadata and indexes are saved.
  }
}

\subsection{diff}
This command retrieves changes between two datapaths.

\texttt{\$ \appcmd diff <oldpath> <newpath> [-s STAGINGDIR] [-o OUTDIR] [-a ANALYZER] [--datachange] [-e {default,threaded,mpi,tigres}]}

The arguments to the command are:

\fbox{
  \parbox{\textwidth}{    
$-$s STAGINGDIR          : directory where filesystem metadata and indexes are saved. \\
$-$o OUTDIR              : directory where the summary of changes is saved.\\
$-$p plugin              : user-defined scripts for analyzing data changes. \\
$--$detailed             : calculate data level changes in addition to file changes.\\
$-$e {default,threaded,tigres,mpi} : type of executor (or runtime) for parallel data change capture.
                         The options are default/threaded/mpi/tigres. The default option uses single
                         threaded. The threaded option uses the Python multiprocessing module
                         that is suitable for parallelizing on one node. For multi-node parallelism,
                         users can select between MPI or tigres.
  }
}

In addition to the four commands, \systemname also provides
two additional commands for cleanup and metadata management.

\subsection{clean}
This option removes all the indexes and cache information
associated with the specified directories.

\texttt{\$ \appcmd clean <path> [path ...]}

The arguments to the command are:

\fbox{
  \parbox{\textwidth}{    
path               : path to data directories.
  }
}

\subsection{metadata}
This command allows users to add user-defined metadata for a data directory. 

\texttt{\$ \appcmd metadata [-m METADATA] [-s STAGINGDIR] {insert,retrieve,append} datapath}

The arguments to the command are:

\fbox{
  \parbox{\textwidth}{    
$-$s STAGINGDIR          : directory where filesystem metadata and indexes are saved. \\
$-$m METADATA            : user-defined metadata information. \\
{insert,retrieve,append} : options related to user-defined metadata information. \\
datapath                 : path to the data directory.
  }
}

\section{Outputs}
\lstset{%
  language=[ISO]C++,
  basicstyle=\ttfamily\footnotesize,
  frame=lines,
  keywordstyle=\color{black},
  commentstyle=\color[rgb]{0.0,0.4,0.0}\scriptsize,
  extendedchars=true,         
  breaklines=true             
}

\systemname prints the summary of changes on standard output. The summary
lists the number of changes between two datasets. An example output looks like below:

\begin{lstlisting}
Added: 1, Deleted: 1, Modified: 1, Metadata-only: 0, Unchanged: 1
\end{lstlisting}

You can opt to save a more detailed output by specifying the output
directory where the detailed change information will be saved:

\texttt{\$ \appcmd diff /old/path /new/path -o output}

The \texttt{output/} directory contains a list of files with detailed
information about the changes. It also contains a summary of the change
information as:

output/summary

\if0
\fbox{
  \parbox{\textwidth}{
change\_factor: 1.0\\
counts:\\
  added: 1\\
  deleted: 1\\
  metaonly: 0\\
  modified: 1\\
  unchanged: 1\\
versions:\\
  base:\\
    dataset\_id: /path/to/old/data\\
    nfiles: 3\\
  revision:\\
    dataset\_id: /path/to/new/data\\
    nfiles: 3
  }
}
\fi

\begin{lstlisting}
counts:
  added: 1
  deleted: 1
  metaonly: 0
  modified: 1
  unchanged: 1
versions:
  base:
    dataset_id: /path/to/old/data
    nfiles: 3
  revision:
    dataset_id: /path/to/new/data
    nfiles: 3
\end{lstlisting}
