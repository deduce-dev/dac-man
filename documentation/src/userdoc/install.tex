This chapter describes the steps required to install \systemname.
It requires Python version 3.6 or greater. It is installed
as any other Python package and uses Python \texttt{setuptools}
package. For enabling advanced features in \systemname, additional
packages may need to be installed. You can install these packages
using \texttt{pip}. 

More general information about installing Python packages can be
found in Python's documentation at \url{http://www.python.org/doc/}.

\section{Installing \systemname}
This section describes the steps for installing \systemname. Upon
successful installation, you can use \systemname as a command-line
tool. 

\subsection{Getting the Package}
You can download the package from the \systemname repository
(\url{https://github.com/dghoshal-lbl/dac-man}) as a tarball.
Alternatively, you can clone the source tree from the repository:

\texttt{\$ git clone git@github.com:dghoshal-lbl/dac-man.git}

\subsection{Installation from Source}
Once the package is downloaded/cloned, \systemname can be installed
by running the following commands:

\texttt{\$ cd dacman}\\
\texttt{\$ python setup.py install}

If you are installing to a location that requires special permissions
(like /usr/local), you may need to run the last command with \textbf{sudo}.
Alternatively, you can create and activate a build environment through
\textbf{virtualenv} or \textbf{conda} as described below.

\subsubsection{Using virtualenv}
You can install virtualenv using pip. 

\texttt{\$ pip install virtualenv}

More details on installing and using virtualenv can be found in
\url{https://packaging.python.org/guides/installing-using-pip-and-virtualenv/}.

After installing virtualenv, you need to create and activate the
environment, and then install \systemname.
 
\texttt{\$ virtualenv venv}\\
\texttt{\$ source venv/bin/activate}\\
\texttt{(venv)\$ cd dacman}\\
\texttt{(venv)\$ python setup.py install}

\subsubsection{Using conda}
Conda can be installed using the OS-specific installer that can be downloaded
from \url{https://conda.io/docs/user-guide/install/index.html}. After installing,
the Python environment can be created and activated as:

\texttt{\$ conda create --name env}\\
\texttt{\$ source activate env}\\
\texttt{(env)\$ cd dacman}\\
\texttt{(env)\$ python setup.py install}

More information about using conda environments can be found in
\url{https://conda.io/docs/user-guide/tasks/manage-environments.html}.

%%%%%%%%%%%%%%%%%%%%%%%%%%%%%%%%%%%%%%%%%%%%%%%%%%%%%%%%%%%%%%%%%%%%%%%%
\section{Testing the Installation}
%\fix{Will add a test suite similar to madats release.}
In order to test the \systemname installation, run the following commands:

\texttt{\$ cd examples/}\\
\texttt{\$ ./scripts/simple.sh}\\

On successful execution, this prints the summary of change and detailed
change information between two example directories.

%%%%%%%%%%%%%%%%%%%%%%%%%%%%%%%%%%%%%%%%%%%%%%%%%%%%%%%%%%%%%%%%%%%%%%%%
\section{Dependencies}
\systemname primarily depends on the following packages:
\begin{itemize}
\item scandir
\item six
\item PyYAML
\item straight.plugin
\end{itemize}

These dependencies are listed in \texttt{requirements.txt} file
and are automatically installed during the build process.

Additional dependencies for running \systemname on HPC environments
include:
\begin{itemize}
\item numpy  : Python library for operations on large, multi-dimensional arrays
\item mpi4py : Python MPI bindings
\end{itemize}

