The plug-ins in \systemname are different `comparators' that capture
changes in files and datasets. All the comparator plug-ins, registered
to \systemname are derived from the base \texttt{Comparator} class.

\section{class Comparator}
The \texttt{Comparator} class is an abstract class that provides
the underlying methods for defining the metadata and algorithm for
capturing data changes. All the methods in this class are abstract
and need to be implemented by the deriving plug-in class.

\subsection{supports()}

A static method that sets up all the file/data types that are
supported by the specific plug-in.

\subsection{description()}

A static method that describes the specific plug-in.

\subsection{compare(a, b, *args)}

The core method for implementing the algorithm to compare two
files/datasets.

\subsection{percent\_change()}

Method to summarize the amount of change in two files with respect
to the comparison algorithm.

\subsection{stats(changes)}

Method for providing statistics on calculated changes from the 
\texttt{compare} method.

\section{Creating Plug-ins}

In order to create and register plug-ins in \systemname, users
need to implement the methods in the \texttt{Comparator} class.
Users can either choose to add multiple plug-ins in a single
module, or create a module for each plug-in.

The following example shows a module for each plug-in (single\_plugin.py):
\begin{lstlisting}[language = python]
class MyPlugin(Comparator):
  def supports():
    ...
  ...
  def compare(a, b, *args):
    ...
  ...
\end{lstlisting}


The following example shows multiple plug-ins in a module (multiple\_plugins.py):
\begin{lstlisting}[language = python]
class TxtPlugin(Comparator):
  def supports():
    return ['txt']
  ...
  def compare(a, b, *args):
    ...
  ...


class CsvPlugin(Comparator):
  def supports():
    return ['csv']
  ...
  def compare(a, b, *args):
    ...
  ...
\end{lstlisting}
