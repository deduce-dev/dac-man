\documentclass[oneside,11pt]{memoir}

\usepackage{graphicx}
\usepackage{url}
\usepackage{color}
\usepackage{subfig}
\usepackage{amsmath}
%\usepackage{algorithm}
\usepackage{listings}
%\usepackage{algpseudocode}
%\usepackage[normalem]{ulem}
\usepackage{xspace}
%\usepackage[skip=0pt]{caption}
\usepackage{multirow}
%\usepackage[most]{tcolorbox}
\usepackage{balance}
%\usepackage{fancyhdr}
\usepackage{courier}

\newcommand{\fix}[1]{\textcolor{red}{#1}}
\newcommand{\systemname}{\textsc{Dac}-\textsc{Man}\xspace}
\newcommand{\appcmd}{dacman\xspace}

\chapterstyle{ell}
%\chapterstyle{hangnum}
%\renewcommand\tocheadstart{}
%\renewcommand\printtoctitle[1]{}

\setlength\parindent{0pt}
\setlength{\parskip}{6pt}
\setsecnumdepth{subsection}
\maxtocdepth{subsection}

\pagestyle{plain}

\pagenumbering{roman}
\begin{document}

\title{\Huge \systemname Documentation \newline\newline \Large Creating Plug-ins: version 1.0.0}

\maketitle

\newpage

%\chapter*{\contentsname}
\tableofcontents*

\chapter{Introduction}
\label{ch:intro}
\pagenumbering{arabic}
\section{Overview}
Scientific datasets are updated frequently due to changes in
instrument configuration, software updates, quality assessments
or data cleaning algorithms. However, due to the large size and
complex data structures of these datasets, existing tools either
do not scale or are unable to generate meaningful change information.

The \textbf{\systemname} (\textbf{DA}ta \textbf{C}hange \textbf{Man}agement)
framework allows users to efficiently and effectively identify,
track and manage data change and associated provenance in scientific
datasets. There are two main components of \systemname:
\begin{itemize}
\item Change tracker that keeps track of the changes between
 different versions of a dataset.
\item Query manager that manages data change related queries.
\end{itemize}

%%%%%%%%%%%%%%%%%%%%%%%%%%%%%%%%%%%%%%%%%%%%%%%%%%%%%%%%%%%%%%%%%%%%%%%%%%%%%%
\section{Features}
The key features of \systemname include:
\begin{itemize}
\item HPC support. \systemname provides MPI support for enabling parallel
 change capture in HPC environments.
\item Offline comparison. Datasets can be compared away from the actual
 location of the data, allowing users to find changes without necessarily
 moving the datasets to a common location.
\item Extendable. Users can plug-in their own scripts to calculate changes.
\item Flexible command-line options. Provides different options to configure
 change detection.
\item Detailed output. \systemname outputs contain details on the different
 types and amount of change.
\item Customizable logging. Users can customize where and what to log,
 including the detailed steps in the change capture process.
\end{itemize}

%%%%%%%%%%%%%%%%%%%%%%%%%%%%%%%%%%%%%%%%%%%%%%%%%%%%%%%%%%%%%%%%%%%%%%%%%%%%%%
\section{Requirements}
\systemname is developed using Python. It requires Python 2.7 or greater.
Users need Python setuptools and pip to install \systemname. Fore detailed
instructions on the installation, please refer to Chapter~\ref{ch:install}.

\systemname is known to work on the following operating systems:
\begin{itemize}
\item Linux
\item Unix-like OSs
\item Mac OS
\end{itemize}


\chapter{The Plug-in API}
\label{ch:api}
The plug-ins in \systemname are different `comparators' that capture
changes in files and datasets. All the comparator plug-ins, registered
to \systemname are derived from the base \texttt{Comparator} class.

\section{class Comparator}
The \texttt{Comparator} class is an abstract class that provides
the underlying methods for defining the metadata and algorithm for
capturing data changes. All the methods in this class are abstract
and need to be implemented by the deriving plug-in class.

\subsection{supports()}

A static method that sets up all the file/data types that are
supported by the specific plug-in.

\subsection{description()}

A static method that describes the specific plug-in.

\subsection{compare(a, b, *args)}

The core method for implementing the algorithm to compare two
files/datasets.

\subsection{percent\_change()}

Method to summarize the amount of change in two files with respect
to the comparison algorithm.

\subsection{stats(changes)}

Method for providing statistics on calculated changes from the 
\texttt{compare} method.

\section{Creating Plug-ins}

In order to create and register plug-ins in \systemname, users
need to implement the methods in the \texttt{Comparator} class.
Users can either choose to add multiple plug-ins in a single
module, or create a module for each plug-in.

The following example shows a module for each plug-in (single\_plugin.py):
\begin{lstlisting}[language = python]
class MyPlugin(Comparator):
  def supports():
    ...
  ...
  def compare(a, b, *args):
    ...
  ...
\end{lstlisting}


The following example shows multiple plug-ins in a module (multiple\_plugins.py):
\begin{lstlisting}[language = python]
class TxtPlugin(Comparator):
  def supports():
    return ['txt']
  ...
  def compare(a, b, *args):
    ...
  ...


class CsvPlugin(Comparator):
  def supports():
    return ['csv']
  ...
  def compare(a, b, *args):
    ...
  ...
\end{lstlisting}


\chapter{Default Plug-in}
\label{ch:default}
The default plug-in in \systemname uses a data abstraction, called
the \systemname records to compare changes for different file formats
and data types. \systemname records transforms the header and data
of a file into an array. The records from two files are then compared
using a linear algorithm.

\section{Adaptors}

The default plug-in uses several adaptors to transform data from
different file formats into \systemname records. Currently, the
plug-in uses adaptors for the following file formats:
\begin{itemize} 
\item h5: HDF5 file format
\item fits: image file format consisting of n-dimensional arrays or tables
\item edf: time-series data
\item tif: high-quality graphics image format
\end{itemize} 

In order to use the default plug-in, users need to install the
required libraries/packages. Following is the list of all the
packages required based for the corresponding file format support.

\begin{itemize} 
\item h5: h5py
\item fits: astropy
\item edf: fabio
\item tif: fabio
\end{itemize}

\section{Change Stats}

The default plug-in captures the following data change metrics:
\begin{itemize} 
\item added:     number of data values added
\item deleted:   number of data values deleted
\item modified:  number of data values modified
\item unchanged: number of unchanged data values 
\end{itemize}

In addition, the default plug-in also calculates \% change
between two files based on the \systemname records comparison.





\chapter{HDF5 Plug-in}
\label{ch:hdf5}
\input{hdf5}

\chapter{Tabular Data Plug-in}
\label{ch:tabular}
\input{tabular}
                                   
\end{document}
