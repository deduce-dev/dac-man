\fix{Fix this later based on the results}
EHPC was built for on-demand elasticity of nodes to be utilized by a job in traditional HPC settings. The nature of the approach taken, with a checkpoint restart mechanism, has shown the benefits inherent in the both elasticity and checkpoint restart. Elasticity in synthetic application scenarios has shown to offer resource savings of up to eighty percent and, in the case of Montage, up to thirty five percent. The uses for EHPC in a batch queue scenario are numerous in both community and added features. 

The results derived from running BLAST with EHPC are characteristic of the benefits obtainable if looking for on-demand adaptation of cores to an application. The scenarios chosen to run BLAST were tailored for a situation in which a user may not know the size of a stream of data, and begins analysis, but once a larger flux of data is observed, the application adapts to the appropriate number of resources.  This can be utilized as a method for adjusting the analysis of a data stream without killing and starting over.  Instead, an application can continue to run in a constrained set of resources until the new cores become avaible.  Once the new set of cores are available, the overhead experienced contains only application checkpoint on the constrained resources and checkpoint and restart overhead on the new set of resources. 

Montage's focus was built around conserving resources of a known widely adopted application. This was shown to be possible by checkpointing Montage once the parallel stages were complete, where resource use contracts greatly near the end of the application.  With EHPC it was shown that resource conservation through removing cores was improved by up to [disputed due to montage error on 12/6].

The immediate benefits that users can obtain through EHPC is the development of an incorporated stable checkpointing system, elasticity of cores available, conservation of core hours, and insight into the landscape of resources utilized by applications. 
