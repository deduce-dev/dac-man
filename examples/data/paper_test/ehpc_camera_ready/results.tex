

%\if0
%\begin{table}[]
%  \centering
%  \caption{Collected Metrics for each experiment}
%  \label{tab:metrics}
%  \begin{tabular}{|c|c| }
%    \hline
%    \textbf{Metric (unit)}                                                       & \textbf{Description}                                                                                                                                                                                                                                             \\ \hline
%    \textit{\begin{tabular}[c]{@{}c@{}}Stage Execution \\ Time (s)\end{tabular}} & Execution run-time for a workflow stage.                                                                                                                                                                                                                         \\ \hline
%    \textit{\begin{tabular}[c]{@{}c@{}}Checkpoint \\ Time (s)\end{tabular}}      & \begin{tabular}[c]{@{}c@{}}Time to checkpoint a stage. It is calculated as the \\ difference between when the Stage Job finishes \\ and when the checkpoint call is triggered.\end{tabular}                                                                      \\ \hline
%    \textit{\begin{tabular}[c]{@{}c@{}}Restart \\ Time (s)\end{tabular}}         & \begin{tabular}[c]{@{}c@{}}Time necessary to restart a workflow stage. \\ It is calculated as the difference between the \\ time when a stage starts and the time when the \\ job related to stage starts.\end{tabular}                                          \\ \hline
%    \textit{\begin{tabular}[c]{@{}c@{}}Process Kill \\ Time (s)\end{tabular}}    & \begin{tabular}[c]{@{}c@{}}Total time needed to kill all processes in all \\ compute nodes. It is calculated only for Contractions\\  as the difference between the time a Contract \\ command is issued and the time a stage \\ checkpoint starts.\end{tabular} \\ \hline
%    \textit{Queue Time (s)}                                                               & \begin{tabular}[c]{@{}c@{}}Total time a workflow job stage waits before \\ starting its execution.\end{tabular}                                                                                                                                                  \\ \hline
%  \end{tabular}
%\end{table}
%\fi

\newcommand{\montageFirst}{\textit{1-Parallel}\xspace}
\newcommand{\montageSecond}{\textit{2-Sequence}\xspace}
\newcommand{\montageThird}{\textit{3-Parallel}\xspace}
\newcommand{\montageFourth}{\textit{4-Sequence}\xspace}

\newcommand{\blastFirst}{\textit{1-Parallel}\xspace}
\newcommand{\blastSecond}{\textit{2-Sequence}\xspace}

\newcommand{\synFirst}{\textit{1-Sequence}\xspace}
\newcommand{\synSecond}{\textit{2-Parallel}\xspace}

In this section, we evaluate the performance and resource usage of
scientific workflows through \systemname. We compare our results
against running the workflows without \systemname (i.e., they are
submitted through a single large job -- see Figure~\ref{fig:ehpc_overview}a). 

\vspace{-0.3cm}
\subsection{Systems} 
We evaluated the impact of elasticity on HPC workflows through
\systemname on two systems -- i) Gordon and ii) Cori.  Gordon~\cite{gordon} is a
dedicated XSEDE cluster with 1024 compute nodes.  Each compute node
contains two 8-core 2.6 GHz Intel EM64T Xeon E5 (Sandy Bridge)
processors and 64 GB of DDR3 RAM. The file system is Lustre with a peak
I/O bandwidth of 100 GB/s and the
resources are managed by TORQUE. Cori~\cite{cori} is a Cray XC40 supercomputer
hosted at the National Energy Research Scientific Computing Center
(NERSC), which has 2388 compute nodes, each with two sockets and
32-core Intel Xeon "Haswell" processor at 2.3 GHz per socket and 128
GB DDR4 memory (2133 MHz, four 16 GB DIMMs per socket). The file
system used during job execution is a Lustre file system with
a peak performance of $~$ 700 GB/s.

%\systemname is executed on the login node, where it creates and submits
%job scripts using \systemname pre-defined templates according to the
%scheduler used in each site, TORQUE in Gordon and Slurm in Cori.
%All \systemname and non\systemname files reside in the Lustre Parallel File System.

\vspace{-0.3cm}
\subsection{Workflows}
We use two real science workflows (Montage and BLAST) and one synthetic
workflow to evaluate \systemname (Figure~\ref{fig:workflow_dags}). All the workflows
are built using the Tigres templates. The different stages in the workflows
are logically grouped together into parallel and sequential stages based
on the resource requirements at each stage of the workflow. 
%Below, we provide a brief overview of the workflows we evaluate.
%in each of the stage types, Parallel or Sequential (see rectangles in Figure \ref{fig:workflow_dags}).

\noindent \textbf{Montage} \cite{jacob2009montage} is an I/O intensive
workload \cite{juve2013characterizing} that constructs a JPEG image from
sky survey data formatted as Flexible Image Transport System (FITS) files
\cite{pence2010definition}. As shown in Figure \ref{fig:montage_dag},
Montage is composed of nine stages, and we logically group them into
four stages -- i) \montageFirst, ii) \montageSecond, iii) \montageThird, and iv) \montageFourth.
%and has different resource requirements in each of them. 
%There are four logical stages labeled \montageFirst\footnote{The \textit{'N'} used in this and all other stage labels indicates the number of cores allocated to the referred workflow stage during experiments. Labels' literal number refers to the workflow logical stages and are ordered as depicted in Figure \ref{fig:workflow_dags}.} and \montageThird, which are parallel stages requiring one or more compute nodes, and \montageSecond and \montageFourth, which are sequential stages and require only one node.  
%All input data needs to be in a unique directory before executing the workflow.  
%Each FITS file sizes 1 MB in total and at the end, a total of 55 GB of output data is generated.
All experimental runs of Montage construct the image for survey \textit{M17}
on \textit{band j} and degree 8.0 from \textit{2mass} Atlas images. 

\noindent \textbf{BLAST} is a memory-intensive workflow that matches DNA
sequences against a large sequence database ( $>$ 6 GB). The workflow splits
an input file (a few KBs) into several small files and then uses parallel tasks
to compare the input against the large sequence database. The database is loaded
in-memory on all the compute nodes during the parallel stage. Finally, all the
outputs from the parallel stage are merged into a single file. 
 As shown in Figure \ref{fig:blast_dag},
BLAST is composed of three stages. As the first stage runtime is short, we logically group them into
two stages -- i) \blastFirst and ii) \blastSecond.
BLAST is used to illustrate the resource usage for an use-case where a parallel stage execution
time is substantially larger than the sequential one. 
%The split operation is a
%small stage in the workflow, and is hence, grouped with the parallel were combined in one logical stage
%herein called \blastFirst.  

\noindent \textbf{Synthetic} workflow is composed
of sequence and a parallel stages (Figure \ref{fig:synthetic_dag}).
%, herein called \synFirst and \synSecond, respectively.  
The workflow is written in Python. The memory-intensive version of
the synthetic workflow consists of tasks that do a large number
of memory allocations for over one billion integers, prior to
calculating the values of their sum and multiplication.
% and then many sum and multiplication operations.  
The first stage contains one billion tasks, calculating the sum in sequence,
whereas the second parallel stage contains ten million tasks, calculating the multiplication
in parallel. In contrast to the other two workflows, this workflow is designed to have
a longer sequential stage, followed by a shorter parallel stage.
We also use the memory-intensive version of the synthetic workflow that consists of 10 thousand parallel tasks. Unless otherwise specified, we use the two stage memory-intensive
synthetic workflow for our evaluation, and use the single stage memory-intensive fully parallel workflow for measuring \systemname overheads.

\begin{table}[t]
\centering
%\begin{tabular}{p{0.2\linewidth}|p{0.7\linewidth}}
\begin{tabular}{p{0.33\linewidth}|p{0.6\linewidth}}
\textbf{Metric (unit)} & \textbf{Description} \\
\hline \hline
\textbf{Stage execution time(s)} & Execution time for a workflow stage \\ \hline
\textbf{Workflow runtime(s)} & Workflow end time - Workflow start time \\ \hline
\textbf{Checkpoint time(s)} & Time to checkpoint a stage \\ \hline
\textbf{Restart time(s)} & Time to restart a workflow stage \\ \hline
\textbf{Queue time(s)} &  Time a workflow stage waits in queue prior to execution \\\hline
\textbf{Core-hours used(hrs)} & $\sum$ Task execution time * Number of cores allocated \\
\hline
\end{tabular}
\caption{\small Metrics for evaluation.}
\label{tab:metrics}
\vspace{-0.5cm}
\end{table}

%\subsection{Metrics}
Table \ref{tab:metrics} lists and summarizes the metrics for our experiments.
It includes workflow runtime, stage execution time, stage checkpoint and restart
times, process kill times, core-hours used and queue time (inter-stage queue wait time).
The runtime of a workflow is calculated as the time between the execution
start of the first stage and completion of the last stage of the workflow.
The core-hours measured correspond to the resource allocated for the entire duration, including possibly resources
left unused by a workflow. Wait time values are not included since jobs
do not consume core-hours when they wait. 
\revcomment{In figure 7-10, which version of E-HPC is used - E-HPC or E-HPC fast ?}
In our evaluation, we use \systemname's regular mode, unless otherwise specified. 


%Additionally, we also calculate core-hours used for each workflow execution
%based on the time and resources the workflow used throughout its execution lifespan.

% where it calculated slightly 
%differently for non\systemname and \systemname. \fix{Why is it different?}  
%
%\fix{I don't think we should be mentioning 3600 - i.e.., just to get right unit} 
%
%\fix{Frankly core hour calculation is how much time the workflow used. This should be rewritten more in what it represents and what is included than the exact math.}
%
%\noindent \textbf{non\systemname} - Calculated as the total sum of
%stage times divided by 3600, then multiplied by the number of CPUs
%allocated for the workflow.
%
%\noindent \textbf{\systemname} - It is calculated as in non\systemname only for parallel stages, which is then summed with Core-hour from Sequential stages. Sequential stages are computed differently since they allocate only one node:
%
%\noindent{\textit{\systemname Sequential Stage}} - Calculated as the total sum of sequential stage times divided by 3600 and then multiplied by the number of CPUs in one node (16 for Gordon and 32 for Cori).

%\vspace{-0.3cm}
\begin{figure}
  \centering
  \includegraphics[width=0.43\textwidth]{figs/synthetic-fast.pdf}
  \caption{\small Synthetic (Gordon) - effects of dynamic resource scaling using \systemname (a) Runtime, (b) Core-hours usage. \systemname scales from 64 to 128  cores and achieves better performance than running the workflow over 64 cores. }
%    \caption{\small Synthetic (Gordon). Figure compares the effects of dynamic resource scaling using E-HPC and shows (a) Turn-around time  and (b) Core-hour usage. E-HPC facilitates scaling from 64 to 128 cores and thus is able to achieve better performance than just running workflow at 64 cores. }
  \label{fig:elastic_demand}
  \vspace{-0.4cm}
\end{figure}

\subsection{\systemname Elasticity}

Figure \ref{fig:elastic_demand} shows the benefits of \systemname for
dynamic resource scaling of an application. In this experiment, we run
the synthetic parallel stage on 64 cores, 128 cores and scaling from
64 to 128 cores and we use fast mode to minimize the wait times. The
fast mode allows applications to continue making progress while other
resources are requested. In Figure \ref{fig:elastic_demand}, the \systemname
calculated bar is generated by taking the data from running \systemname in
fast mode and adding the queue time for the second job. We see that
queue times are not substantial in this case. However, \systemname in the fast mode
provides even more significant benefits in cases where queue times
are significant. 


%As shown in Figure \ref{fig:elastic_demand}, 
%the Fast Mode allows for faster turnaround even when queue times 
%are accounted for and checkpoint overhead is substantial. This is due to 
%the ability to mitigate queue time as a factor in turnaround time.

%while utilizing 12 percent more core hours than
%having originally requested 128 cores. 

Figure \ref{fig:elastic_demand}a shows that fast mode results in around 30
percent improvement in workflow runtime as compared to maintaining the
resources at 64 cores. The application is able to benefit from the
added cores and complete the application sooner. Figure
\ref{fig:elastic_demand}b shows corresponding core-hours expenditure
for each run.


%To account for pace of execution, the time spent
%running in the smaller resource set was moved to the second stage and
%normalized to the new resources.  This effectively doubled the pace of
%the application. With these adjustements it is observed that the Fast
%Mode of E-HPC effectively beats the standard version of E-HPC, and
%Fast Mode benefits from longer queue times over the standard E-HPC
%method.






%\vspace{-0.3cm}
\begin{figure*}[htbp]
	\centering
	\subfloat[Runtime.]{\includegraphics[width=0.4\textwidth]{figs/montage_turn-around-new.pdf}
		\label{fig:montage_turnaround-new}}
	\subfloat[Core-hours.]{\includegraphics[width=0.4\textwidth]{figs/montage_core-hour-new.pdf}
		\label{fig:montage_core-hour-new}}
	\caption{\small Montage workflow performance (Cori):
	(a) workflow runtime  and (b) core-hours usage for Montage with and without \systemname. 
	Montage shows shorter runtimes without \systemname.
	For values of n larger than 32, \systemname runs consume less core-hours.
        \revcomment{Reviewer 1 has comments about these results. I am not sure what was meant.}
%	However, the sequence stages use less core-hours in \systemname due to dynamic elastic resource allocation.
	}
% Gonzalo Edit
%\caption{\small \fix{Montage workflow (Cori): (a) workflow runtime  and (b) Core-hours usage for Montage with \systemname and without \systemname.  sequence stages are allocated 1 node (i.e., 32 cores) and use one core. Parallel stages scale linearly, while sequence stages suffer due to I/O after 32 cores. The sequence stages incur lower core-hour expenditure in \systemname. Parallel stages use roughly same core-hours in \systemname and non\systemname cases. However, the sequence stages use less core-hours in \systemname
%due to dynamic elastic resource allocation.}}
	\label{fig:montage_results}
	\vspace{-0.4cm}
\end{figure*}
\subsection{Effect of Stage Elasticity}
%In one of its use cases, \systemname 
%adjusts workflow resources to the static,
%but different requirements of each workflow stage.
%In this case, elasticity is achieved by including a call to
%\systemname API in the workflow,
%before the start of each stage to
%express its resource requirements.
%In this section, performance results of workflows
%run in this use case are presented.

The stage elasticity in \systemname allows workflows to
request resource changes between the stages of a workflow
(as described in Section~\ref{subsec:user}). In this section,
we present the results of using stage elasticity in \systemname
for different workflows. 

In our evaluation, we measure \systemname performance using workflow
runtime and allocated core-hours. Values observed in each experiment
are presented in Figures~\ref{fig:montage_results}
to~\ref{fig:synthetic_results}.  Each bar in the figure presents the
average value (with standard deviation bars) over three repetitions of an
experiment. 

Experiments include runs with and without \systemname and different
resource allocation (32, 64, 128 and 256 cores).
%Y-axis of the figures indicates the measured metric and 
The X-axis represents the peak CPUs (\textit{n}) allocated for the
workflow in a particular experiment.  In non\systemname runs, a value
\textit{n} on the X-axis corresponds to the number of CPU cores
allocated during the complete lifecycle of a workflow. In \systemname
runs, \textit{n} is the maximum number of CPU cores allocated during
the duration of a workflow.

%number of CPU cores allocated to a parallel stages. Sequence stages always
%allocate 32 cores.
%\fix{Movied this under Metrics.}
%The runtime of a workflow is calculated as the time between the execution
%start of the first stage and completion of the last stage of the workflow.


%In the figures, main contributors to the runtimes are decomposed as stacked
%bars including: runtime of each stage (labeled as \emph{[order in the
%execution]-[Sequential/Parallel]}), wait time of each stage if run as an
%independent job (\emph{[order in the execution]-Queue}), and aggregated runtime
%consumed by checkpoint and restart operations (\emph{Checkpointing}).

% moved under metrics
%The core-hours measured in this section correspond to allocation,
%including resources left unused by a workflow. Core-hours are labeled
%and decomposed in stacked bars like the runtime figures. Wait time values
%are not included since jobs do not consume core-hours when they wait.

%Y-axes workflow indicate averagethe runtime expressed  in seconds, with the standard deviation shown by vertical intervals over stage times.  The turnaround time is calculated at the start of first workflow task to end of the workflow execution.  
%For each workflow, we compare the different factors contributing to the total turnaround time, represented by the bars in Figures
%\ref{fig:montage_turnaround-new}, \ref{fig:blast_turnaround-new} and \ref{fig:synthetic_turnaround-new}. 
%These factors are stage executions time, checkpoint, restart, process kill times, and inter job wait times (Queue Time).
%Each bar presents the average workflow execution runtime in seconds with and without \systemname for four allocation sizes \textit{n}: 32, 64, 128 and 256 cores. 
%
%\fix{Pasted from below needs to be integrated.}
%
%In this section, we compare the core-hours used under both execution modes - \systemname and non\systemname for the three workflows as shown in Figures \ref{fig:montage_core-hour-new} \ref{fig:blast_core-hour-new} and \ref{fig:synthetic_core-hour-new}.  
%The X-axis show the peak CPUs allocated in each run and Y-axis shows the total core-hours. 
%We also show the standard deviation with vertical bars over stage times related to the experimental runs.
%Since \systemname uses one node for sequential stages,
%its core-hours is lower than non\systemname in all sequential stages.  
%Each bar in the Figures represents the Core-Hour contributions from each Stage execution runtime and the Overhead times (Checkpoint, Restart and Process Kill times),
%which were summed up and joined together in one stack as they are very small compared to the total core-hour expenditure. 



%In this section, we evaluate and compare the performance and resource usage of
%workflows with and without \systemname as per the different workflow patterns.
\vspace{-0.3cm} 
\subsubsection{Montage}
\begin{figure*}[htbp]
  \centering
  \subfloat[Runtime.]{\includegraphics[width=0.4\textwidth]{figs/blast_turn-around-new.pdf}
    \label{fig:blast_turnaround-new}}
  \subfloat[Core-hours.]{\includegraphics[width=0.4\textwidth]{figs/blast_corehrs-new.pdf}
    \label{fig:blast_core-hour-new}}
  \caption{\small BLAST workflow performance (Cori):
  (a) workflow runtime and (b) core-hours usage for BLAST with \systemname and without \systemname.
  BLAST execution is dominated by its first parallel stage.
  BLAST runtimes and core-hours are similar under both approaches, with slightly longer times
  and higher core-hour numbers in \systemname.}
%  \caption{\small \fix{BLAST workflow (Cori): (a) Total turnaround time and (b) Core-hours usage for BLAST with \systemname and without \systemname. The sequence stages are allocated 1 node (i.e., 32 cores) and use one core. The parallel stage scales linearly, which is the dominant stage in BLAST. The sequence stage is very small due to its short operation. 
%%\fix{What??? -  \systemname contribute to lower core-hour expenditure in \systemname.} 
%Parallel stages scale roughly the same way in \systemname and non\systemname, though \systemname incurs a small overhead due to checkpoint and restart. Core-hours usage does not improve with \systemname due to extrememly small sequence stage. Scaling down incurs a checkpoint/restart overhead that overshadows the scaled down resource allocation for the sequence stage.}}
  \label{fig:blast_results}
  \vspace{-0.4cm}
\end{figure*}

\begin{figure*}[htbp]
  \centering
  \subfloat[Runtime.]{\includegraphics[width=0.4\textwidth]{figs/synthetic_turn-around-new.pdf}
    \label{fig:synthetic_turnaround-new}}
  \subfloat[Core-hours.]{\includegraphics[width=0.4\textwidth]{figs/synthetic_core-hours-new.pdf}
    \label{fig:synthetic_core-hour-new}}
  \caption{\small
  Synthetic workflow performance (Cori): (a) workflow runtime and (b) core-hours usage for the Synthetic workflow with \systemname and without \systemname.
  Synthetic workflows show significantly shorter runtimes without \systemname.
  Until 128 cores, less core-hours are allocated under \systemname.
  For higher values of n, \systemname uses less core-hours.
}
%    \caption{\small \fix{Synthetic workflow (Cori): (a) Total turnaround time and (b) Core-hours usage for the Synthetic workflow with \systemname and without \systemname. The sequence stages are allocated 1 node (i.e., 32 cores) and use one core. Parallel stages scale linearly, but the overall workflow perform porrly in \systemname due to large number of memory allocation/deallocation calls. DMTCP monitors all memory allocation operations, resulting in high runtime overheads in \systemname. \systemname uses larger core-hours for 32 cores, but uses lesser core-hours for increasing cores. Core-hours usage for the parallel stage increases due to larger overheads, but the core-hours usage decreases significantly for the sequence stage because \systemname allocates fewer resources (1 node, 32 cores as compared to 256 cores) for the sequence stage.}}
% In (b), \systemname contribute to lower core-hour expenditure in \systemname. Parallel stages contribute roughly in the same way in \systemname and Non\systemname.} }
  \label{fig:synthetic_results}
  \vspace{-0.4cm}
\end{figure*}
In this section, we evaluate the performance and resource usage of
Montage.

\noindent\textbf{Workflow Runtime.}
Figure~\ref{fig:montage_turnaround-new} compares the workflow runtime for Montage
with and without \systemname. 
%Gonzalo: this is already said before, so I comment it out.
%
%Montage is an I/O intensive workflow and most of the overheads 
%are concentrated in the parallel filesystem, especially for intermediate 
%I/O. 
%with file
%handling happening in between all stages depicted in Figure
%\ref{fig:montage_dag}.  
When running without \systemname, the workflow runtime does not
change substantially across different values of \textit{n} and
the shortest one is observed for $n=32$ (single node on Cori).
%This result is counterintuitive, as it was expected
%that assigning more results to the workflow tasks would
%reduce its runtime.
For larger $n$, the runtime of sequence stages
(\montageSecond and \montageFourth) increases equally or more
than the runtime gains in the parallel ones (\montageFirst and \montageThird).
%Execution analysis associates this phenomenon to changes 
This is likely due to the inter-stage data caching for different values of $n$.
%because sequential stages run always on 
%a node independently of $n$-
For instance, if $n=32$, \montageFirst runs on a single node and all
its output data ($4.5$ GiB) is cached locally (and will eventually be
written to the file system). As a consequence, \montageSecond reads
its input data mainly from memory.  However, for $n=64$, \montageFirst
runs across two nodes, caching one half of its output data on each
node.  When \montageSecond starts on one of the two nodes, only half
of its input data is locally cached.  The runtime of the I/O intensive
stage becomes $15\%$ longer than for $n=32$.
%For larger $n$, the cached data size decreases proportionally
%and the runtime keeps increasing accordingly.
This effect is also observed for \montageFourth since
it is also an I/O intensive sequential stage preceded
by a parallel one (\montageThird).
However, since its input data is larger than for
\montageSecond ($38$ GiB), the effect is more noticeable,
e.g., runtime of \montageFourth from $n=32$ to
$n=64$ increases $52\%$.
%
%
%For instance, stage \emph{mProjDiff} runs over a parallel template, it is I/O
%intensive, and, as expected, its runtime becomes shorter for larger value of $n$.
%\emph{mConcatBg} is a sequential stage, also I/O intensive, and its runtime
%increases significantly from  $n=32$ to $n=64$ and increases slightly
%for larger values of $n$.
%Since \emph{mConcatBg}  is a sequential stage, $n$ does not include
%on the code execution our its output write.
%As a consequence, the increase of runtime of this stage
%must be related to the stage input read.
%The size of the intermediate data was measured, \emph{mConcatBg}
%reads $4.5$ GiB of data, which is significantly smaller than the
%memory of Cori's compute nodes ($128$ GiB).
%However, as $n$ increases, the effect of this caching is different.
%For $n=32$,  \emph{mProjDiff} runs in a single node
% its output cached by the local I/O subsystem, 
%and then read from memory by the \emph{mConcatBg} stage.
%For $n=64$,  \emph{mProjDiff} runs on two nodes, all of its
%output is stored in the file system, but half of it is cached
%in a compute node, and the other half in another.
%The \emph{mConcatBg} stage runs on one of the nodes
%used \emph{mProjDiff}, so $50\%$ of the data is not locally
%cached, which makes the read slower than for $n=32$.
%This explains the large difference in \emph{mConcatBg} runtime 
%between $n=32$ and $n=64$.
%For $n=128$, and $n=256$,
%the locally cached data becomes even smaller 
%($25\%$ and $12.5\%$), 
%which explains the slight increase in runtime for 
%larger values of $n$.
%
%The increase of runtime of \emph{mJPEG} for larger values of $n$ is
%even more significant than for \emph{mConcatBg}.
%This is explained by the I/O nature of  \emph{mJPEG} and 
%the preceding parallel stage and the larger size of
%the intermediate date ($38$ GiB).

%\grfix{Why does it not change? Most people will expect it to go down with increased cores} 
%Parallel
%\emph{mProjDiff} and $mBackground$ stages represent shorter times of the
%total runtime and show linear scaling \grfix{don't understand - due to
%  resource distribution among compute nodes}.  Sequential \emph{mConcatBg}
%and \emph{mJPEG} stages are affected by scaling (64, 128 and 256), where
%sequential \emph{mJPEG} runs longer than 32 cores.  The reason for this
%performance loss in the fourth stage for both scenarios are \grfix{I
%  dont understand this - too vague and hand wavy} due to the way files
%are cached in the I/O buffer among compute nodes.

When run with \systemname, Montage workflow runtime presents
a different pattern. Again, workflows running on a single node ($n=32$)
present the shortest runtime because the tasks run on a single node
and in the same job allocation. For $n>32$, the inter
job wait time increases the workflow runtime significantly compared to $n=32$.
This is expected since for $n>32$, \systemname runs each section of the workflow
in a separate job to adjust the resource allocation to the desired size  (Figure \ref{fig:montage_dag}). 
As $n$ increases, the runtime for the sequence stages does not
change and parallel stages becomes shorter, decreasing the overall runtime.
%This is different from the pattern observed in non-\systemname runs
%because workflow divided between jobs cannot benefit from intermediate data caching,
%as new resources are allocated for each job.

%For \systemname, the runtime doesn't change substantially when
%\textit{n} increases. \grfix{If this is the same for both, we can start
%  with that in both. What is the point of showing n on X axis if it
%  doesnt change?}  Parallel stages \emph{mProjDiff} and $mBackground$ scale
%linearly, similarly to Non\systemname.  Sequential \emph{mConcatBg} and
%\emph{mJPEG} stages are affected by scaling due to the I/O buffering, and
%differences in runtime per stage are low when compared to
%Non\systemname. \grfix{Why? why? why?}  A major contributor for a longer
%turnaround time for \systemname in comparison to Non\systemname were
%Queue \emph{mConcatBg}, $mBackground$ and \emph{mJPEG} times, added in between
%the stages. \grfix{Why?} Workflow-aware schedulers could be used to mitigate this
%behavior \cite{rodrigo2017enabling}. \grfix{How?}
%

The comparison between running Montage with or without \systemname
shows that runtime is longer with \systemname in all cases.
When run on a single node, the 20\% runtime increase is due to
monitoring overhead of DMTCP. For $n>32$, the workflow runtime
difference is contributed by the inter-job wait time and longer runtime of the stages.
The inter job wait time is dependent on the current workflow of the system and out of the
control of \systemname. 
%The stages run longer with \systemname because of 
%DMTCP's monitoring overhead and the impossibility of inter-stage data caching.
For most cases, total stage runtime is $\approx 20\%$
longer with \systemname. As we scale to $n=256$, non\systemname runs can no longer benefit
from inter-stage data caching due to data being distributed across
multiple node, and hence, stage runtime overhead in \systemname is reduced to $10\%$
as compared to non\systemname. 
%The runtime results show that the performance
%of the workflow with \systemname is affected only due to the checkpoint-restart
%overhead and the underlying application characteristics.

\noindent\textbf{Workflow core-hours.}
Core-hours consumed by all the experiments with Montage on
Cori are detailed in Figure \ref{fig:montage_core-hour-new}.
In cases without \systemname, larger allocations
increase the core-hours consumed. Without elasticity, the
sequence stages consume significantly more 
core-hours since their runtime is not reduced by the
larger resource allocation. Also, parallel stages, when
scaling up from $n=32$ to $n=256$, consume
slightly more core-hours due to the increasing overheads
of the initial setup in Tigres for launching the parallel tasks across multiple nodes %due to imperfections.
For Montage, $n=32$ (single node on Cori),
72\% of the workflow runtime is consumed by serial stages
(\montageSecond and \montageFourth). 
%The importance of the sequential stages
%causes the linear increase of core-hours consumed as $n$ grows.

With \systemname, for values of $n>32$, doubling the allocated
resources induces small variations in consumed core hours. For
example, stepping $n$ up from 64 to 128, increase core-hours
consumption by 11\%.  Core-hour usage increases are attributed to the
natural overhead of less than perfect parallelism in the code, and the
initial overhead of distributing the tasks through Tigres across
multiple nodes.  Otherwise, with \systemname, there is no resource
wastage and checkpointing core-hours are very small ($<1\%$).
However, there is a larger step between $n=32$ and $n=64$, with an
increase of 35\% of core hours.  This is due to the loss in efficiency
in the sequence stages due to lack of caching of intermediate data.

Finally, comparing runs of Montage with the two approaches,
for values $n>32$ (elasticity is possible),
\systemname requires significantly less core-hours ($76 \%$ for 256 cores)
than non\systemname due to elastic management of resources.
The core-hour results in Montage show that with increasing
parallelism, \systemname utilizes resources more
efficiently than non\systemname due to diverse level of parallelism.
% for workflows with diverse level of parallelism.
 
% This demonstrate the capacity of the system
% to adjust resource allocations to the workflow
%demand efficiently and to minimzie resources wastage.



%
%
%Non\systemname core-hour expenditure gets worse as the number of CPUs allocated \textit{n} scales out.
%It goes from 1\% worse in 64 cores to up to 400\% worse in 256 cores. 
%It is mainly due to its poor core reservation as one big job, mostly affected in sequential stages (\montageSecond and \montageFourth). 
%Since \systemname sequential stages runtime overheads are lower in comparison to non\systemname, \montageSecond and \montageFourth stages were the ones that negatively affected non\systemname core-hours consumption.
%As \systemname allocates only one node (32 cores) for these stages, it spent less core-hour than non\systemname in all cases.
%


%\grfix{This is hanging and not clear where it fits into the rest of the
%  explanation} In both Non\systemname and \systemname, when scaling
%out to more than 32 cores (one node), the cache miss ratio might
%increase as the workflow tasks are placed in different nodes at each
%stage.  This is important particularly when the workflow transitions
%from a parallel to a sequential stage as these tasks try to access
%files that are not cached in the nodes they were placed, causing many
%I/O cache misses.  Given these cache misses, the requested files have
%to be loaded at runtime, decreasing the I/O performance and increasing
%overall stage runtime.  \\

\vspace{-0.3cm}
\subsubsection{BLAST}
This section focuses on evaluating the impact of \systemname on BLAST.

\noindent\textbf{Workflow runtime.}
The runtime of all the experiments running BLAST 
on Cori are presented in Figure~\ref{fig:blast_turnaround-new}.
When run without \systemname, 
BLAST's runtime is dominated by the first parallel stage  (\blastFirst occupies  $>99\%$
of the total runtime in all cases).
This stage scales well over more resources and overall workflow runtime is significantly reduced when run
over more resources.
For instance, for $n=64$ the runtime is less than half ($53\%$ shorter) than for $n=32$.
%As expected the runtime of the sequential stage (\blastSecond) does not change
%as resources are increased.

Similar workflow runtimes are observed in BLAST when run with \systemname.
DMTCP's monitoring overhead is relatively small but becomes more significant
for larger values of $n$. 
e.g., DMTCP increases BLAST's runtime by $1\%$ for $n=32$ and $8\%$ for $n=256$.
For all values of $n$,
 checkpoint, restart, and queue times increase the overall workflow runtime by 
$\approx 2$ minutes, which is not significant compared to the overall workflow runtime.
In summary, BLAST scales well as the workflow allocation is increased,
and has very little overhead when run with \systemname.

% For non\systemname workflows, \blastFirst stage represents the largest
% portion of total execution time as can be seen in Figure \ref{fig:blast_turnaround-new}.
%\blastFirst tasks are equally distributed among compute-nodes as resources scale,
%equally reducing the load in each compute node and explaining its linear scalability.  
%On the other hand, \blastSecond stages are unaffected by scaling, since
%only one compute node is necessary to execute its simple file merge operation.
%
%The comparison between running BLAST with or without \systemname
%shows that runtime is longer with \systemname in all cases.
%For \systemname, similarly to non\systemname, results shown stage
%runtime decreases in half when the number of nodes \textit{n}.
%Extra overheads are due to checkpoint, restart and queue times.
%Excluding overheads, stage runtimes show low overhead in comparison to non\systemname. 
%This is so because BLAST compares DNA sequences with the sequence
%database residing in memory.
%This type of operation does not incur in any substantial overhead to \systemname, 
%running under DMTCP, enabling it to have a nearly zero stage runtime footprint in relation to non\systemname.
%Finally, $Sequence Merging$ stages are short and unaffected by scaling,
\begin{figure*}[htbp]
  \centering
  \subfloat[Runtime.]{\includegraphics[width=0.4\textwidth]{figs/blast_execution_new.pdf}
    \label{fig:blast_execution}}
  \subfloat[Core-hours.]{\includegraphics[width=0.4\textwidth]{figs/blast_charged_new.pdf}
    \label{fig:blast_charged}}
  \caption{\small Runtime elasticity on BLAST (Gordon) vs static allocation: (a) workflow runtime and (b) core-hour usage for BLAST. The \systemname coordinated job starts on a single 16 core node and expands to the peak core allocation after 60 seconds of execution. }
%    \caption{\small Runtime elasticity on BLAST (Gordon): Workflow turnaround time (a) and Core-hour usage (b) versus maximum cores utilized for BLAST with \systemname and without \systemname. The \systemname coordinated job starts on a single 16 core node and expands to the Peak Core Allocation after 60 seconds of execution. }
  \label{fig:blast_gordon}
    \vspace{-0.4cm}
\end{figure*}


\begin{figure*}[htbp]
  \centering
  \subfloat[Performance overhead]{\includegraphics[width=0.4\textwidth]{figs/time_dmtcp.png}
    \label{fig:time_dmtcp}}
  \subfloat[Checkpoint overhead]{\includegraphics[width=0.4\textwidth]{figs/size_dmtcp.png}
    \label{fig:size_dmtcp}}
  \caption{\small \systemname overheads: (a) shows time required for checkpoint and
    restart vs. the number of processes/tasks being tracked by \systemname, (b) shows total
    storage required on filesystem for a checkpoint versus the number of processes/tasks
    being tracked by \systemname.}
  \label{fig:overheads}
    \vspace{-0.4cm}
\end{figure*}

\noindent\textbf{Workflow core-hours.}
Core-hours consumed with BLAST (on Cori) are detailed in Figure \ref{fig:blast_core-hour-new}. 
Similar core-hours are observed 
with and without \systemname and different values of $n$, e.g., the maximum %and 
values differ less than 9\% from the average.
This is caused by the domination of (\blastFirst)
over the execution of the workflow that makes other
non-scaling stages irrelevant in terms of core-hours.
%As observed in Figure~\ref{fig:blast_turnaround-new},
%this parallel stage runtime decreases significantly
%when its parallelism is increased.

The comparison between $n=32$ and $n=64$ 
cases present an unexpected result: the core-hours are reduced
when parallelism is increased.
This is caused by the unexpected super-linear reduction
of runtime in that step of $n$ observed in Figure~\ref{fig:blast_turnaround-new}.
%\fix{I really liked these statements, but the language sounds very philosophical.}
In the next steps of $n$, core-hours increase slowly
from the expected imperfection of parallelism in the code.

Comparison between using and not using \systemname
%Comparison between \systemname and non\systemname
shows that% for all values of $n$, 
\systemname consumes
1.5\% to 5\% more core-hours with no clear correlation to $n$. 
Checkpointing overhead in all cases consumes
less than 0.5\%.
This leads to the conclusion that
the additional core-hours consumed by \systemname
are for DMTCP execution overhead.


%We can see \systemname shows a small overhead.
%\blastFirst stage in BLAST is parallel and \systemname had a low execution runtime overhead, thus making Non\systemname spend roughly the same core-hours as \systemname.
%Core-hours are then mainly dominated by the Parallel \blastFirst stages.


%as it is a file merge operation.

\vspace{-0.25cm}
\subsubsection{Synthetic}
This section describes our evaluation of the Synthetic workflow with \systemname.

\noindent\textbf{Workflow runtime.} 
The runtimes observed of all experiments with Synthetic workflow are presented
in Figure~\ref{fig:synthetic_turnaround-new}.
%The Synthetic workflow was used for
%illustrating a memory-allocation-intensive workload.
When run without \systemname, the workflow runtime becomes shorter 
for larger values of $n$.
This reduction is the result of shorter stage runtimes
of the \synSecond stage when more resources 
are available (\synFirst is sequential and thus its
runtime is constant): \synSecond  runtime
is reduced $40-49\%$ each time $n$ is doubled.

The Synthetic runs with \systemname present much
longer runtimes than without \systemname.
This is due to  %very important 
DMTCP monitoring overhead that slows down 
the execution of all stages by a $2.2-2.4$ factor.
Detailed analysis of the workflow reveals 
that most of the operations performed by the workflow
were memory management
(allocation and free). 
These operations are heavily monitored
by DTMCP that traps all the memory management calls. %The stage executions are consequently, slowed down.
The workflow runtime evolution
for larger values of $n$ is as expected:
\synFirst runtimes remain constant, and
\synSecond runtime is reduced significantly
(again  $40-49\%$). 
Finally, \systemname checkpoint overheads 
are minimal (6 seconds for all values of $n$)
and the queue times for the second job is typically a few minutes.

\noindent\textbf{Workflow core-hours.}
Core-hour consumed in all experiments with Synthetic
are detailed in Figure \ref{fig:synthetic_core-hour-new}. 
For  non\systemname, a larger resource allocation
implies a significant increase in core-hours
consumed by the workflow.
This increase is mainly due to 
the wastage of the resources by the \synFirst stage. 
%The \synSecond also consumes a bit more resources for
%larger values of $n$, but the increase is not significant.

Runs of Synthetic with \systemname consume
almost the same core-hours for all values of $n$.
This is caused by the constant resource consumption
of both stages in the workflow.
\synFirst consumes the same core hours because
elasticity % provided by the system
allows to execute \synFirst over 32 cores in all cases.
\synSecond  runtime decreases proportionally to
the increase in assigned resources, keeping
its core-hours consumption almost unchanged.

%Comparison between \systemname and non\systemname
%cases show a case of break-even.
%For values of $n<128$ non\systemname
%consume less core-hours.
%This is because Synthetic workflow is memory-allocation-intensive and
%suffers a great runtime overhead when executed with DMTCP.
%However, as $n$ increases, the wasted core-hours 
%in \synFirst with non\systemname
%become larger than the DMTCP overhead,
%and \systemname becomes more efficient. % in terms of core-hours.


%For non\systemname we see high increases in \synFirst stage core-hour usage due to its sequential operations.
%As the number of cores (X axis) is allocated for whole execution, only the \synSecond stage could effectively use all of them.
%In \systemname we see nearly constant total core-hour usage for all
%cases since all \synFirst stages used only one node and
%\synSecond stages scaled linearly.  
%These two compensating factors help \systemname having lower core-hour usage starting at 128 cores.
%Even though runtimes for both stages in \systemname were $2.4x$ higher (see previous subsection), the number of cores used was more efficient, especially in the \synFirst parts, which are larger than \synSecond stages.  
%This shows a use case where even though \systemname had a higher runtime overhead, it would still make efficient resource usage in relation to core-hours.  
%Surprisingly, this is not always the case due to the high overhead introduced by DMTCP which tracks all memory allocations done in this workflow (explained in the previous subsection). 
%
%Core-hours for non\systemname and \systemname are detailed in Figure \ref{fig:synthetic_core-hour-new}. 
%For non\systemname we see high increases in \synFirst stage core-hour usage due to its sequential operations.
%As the number of cores (X axis) is allocated for whole execution, only the \synSecond stage could effectively use all of them.
%In \systemname we see nearly constant total core-hour usage for all
%cases since all \synFirst stages used only one node and
%\synSecond stages scaled linearly.  
%These two compensating factors help \systemname having lower core-hour usage starting at 128 cores.
%Even though runtimes for both stages in \systemname were $2.4x$ higher (see previous subsection), the number of cores used was more efficient, especially in the \synFirst parts, which are larger than \synSecond stages.  
%This shows a use case where even though \systemname had a higher runtime overhead, it would still make efficient resource usage in relation to core-hours.  
%Surprisingly, this is not always the case due to the high overhead introduced by DMTCP which tracks all memory allocations done in this workflow (explained in the previous subsection). 
%
%the runtime of \synFirst is the same
%for all values of $n$ ($\approx 600s$),
%
%For non\systemname, there is no scaling in \synFirst stages since its tasks are executed in only one node.  
%For \synSecond stages, linear scaling is seen as
%all task operations are equally divided among available compute nodes.
%
%The comparison between running Synthetic with or without \systemname shows that runtime is greatly longer for \systemname in all cases.
%For \systemname, there is no scaling in \synFirst stages, similarly to
%non\systemname.  
%For \synSecond stages we see again a linear scaling in all
%cases.  
%Checkpoint and Restart times are negligible and it is due to
%the small code memory footprint in the synthetic process.  
%Finally, queue times were different among runs as they vary based on system utilization \cite{nurmi2006evaluation, nurmi2007qbets}. 
%
%The stage runtime overhead in the Synthetic workflow increases significantly when run with \systemname. 
%For all experiments, sequential stage runtimes (\synFirst) more than doubled under \systemname.
%Differently to other workflows, the Synthetic workflow has
%no I/O and is fully written in Python.
%Also, the main code running during most of its execution
%is composed by a loop performing mathematical operations.
%However, each loop iteration includes the creation of two list structures that are immediately discarded after the mathematical operation.
%This forces two memory allocation system calls per iteration,
%which, under \systemname, are inspected and slowed down by DMTCP.
%Thus, the Synthetic workflow is memory allocation intensive
%since, in each stage, it creates and immediately
%destroys 5.6 billion Python list objects. 
%We can then conclude that runtime overheads will be very significant
%for application intensive in memory allocation operations.


% Gonzalo: We cannot explain the overhead in the synthetic workflows
% in detail. By now, I trim this out.
%
%\fix{This paragraph is hanging and does not flow with the above and
%  the explanation comes too late} \fix{which- This} high overhead in
%the Synthetic workflow is because its code calls the Python built-in
%$sum()$ function in each of its stage tasks, either in $s-Sequential$
%and in $s-Parallel$.  This function is implemented in C and is called
%through Cython, and thus it is an external call provided through a
%shared library outside to the Tigres Python process itself. \grfix{C
%  calls should be faster so it is hard to believe this the way it is
%  written} DMTCP tracks all system calls a user program does, except
%file handler ones (as was the case for most of Montage and BLAST
%operations).  Calling a function in a shared object triggers a system
%call, thus triggering the DMTCP tracking system, which decreased
%compute performance due to its complex internal logic for providing
%checkpoint-restart mechanisms \cite{dmtcp_benchmarks}.  The $sum()$
%function was called $28*10^8$ times in the $s-Sequential$ stage and
%$11*10^7$ in the $s-Parallel$ stage.  Each one of these calls are
%registered by the DMTCP tracking system, increasing the runtime
%overhead for both stages. \grfix{this is both too detailed and too
%  little detail - hard for me to understand really what is going on}

%%%%%%%%%%%%%%%%%%%%%%%%%%%%%%%%%%%%%%%%%%%%%%%%%%%%%%%%%%%%%%%%%%%%%%%%%




%\iffalse
%\begin{figure}
%	\centering
%	\includegraphics[width=3.18in]{figs/blast_turn-around-new.pdf}
%	\caption{Total turn-around time versus maximum cores utilized for BLAST with \systemname and without \systemname. Parallel stages scale linearly, while Sequential are very small due to its short operation. \systemname's sequential stages use 32 cores. }
%	\label{fig:blast_turnaround-new}
%\end{figure}
%
%\begin{figure}
%	\centering
%	\includegraphics[width=3.18in]{figs/blast_corehrs-new.pdf}
%	\caption{Core-hour versus maximum cores utilized for BLAST with \systemname and without \systemname. \systemname's sequential stages use 32 cores and contribute to lower core-hour expenditure in \systemname. Parallel stages contribute roughly in the same way in \systemname and Non\systemname.  }
%	\label{fig:blast_core-hour-new}
%\end{figure}
%\fi



%\iffalse
%\begin{figure}
%	\centering
%	\includegraphics[width=0.5\textwidth]{figs/synthetic_turn-around-new.pdf}
%	\caption{Total turn-around time versus maximum cores utilized for Synthetic with \systemname and without \systemname. Parallel stages scale linearly, while Sequential stages are within same intervals for both cases. \systemname's sequential stages use 32 cores. }
%	\label{fig:synthetic_turnaround-new}
%\end{figure}
%
%\begin{figure}
%	\centering
%	\includegraphics[width=0.5\textwidth]{figs/synthetic_core-hours-new.pdf}
%	\caption{Core-hour versus maximum cores utilized for Synthetic with \systemname and without \systemname. \systemname's sequential stages use 32 cores and contribute to lower core-hour expenditure in \systemname. Parallel stages contribute roughly in the same way in \systemname and Non\systemname.  }
%	\label{fig:synthetic_core-hour-new}
%\end{figure}
%\fi


%\vspace{-0.3cm}



%\begin{figure*}
%  \centering
%  \subfloat[Turn-around time.]{\includegraphics[width=0.5\textwidth]{figs/wastage_new_runtime.pdf}
%    \label{fig:synthetic_time_gordon}}
%  \subfloat[Core-hour usage.]{\includegraphics[width=0.5\textwidth]{figs/wastage_new_core-hours.pdf}
%    \label{fig:synthetic_corehrs_gordon}}
%  \caption{\small \fix{This graph is actually showing nothing- 1. This is not for runtime elasticity, 2. The non-EHPC version ran under DMTCP, but without checkpointing; so had all the overheads of DMTCP monitoring and system call traps} Effect of runtime elasticity on synthetic workflow: The
%    total core hour requirements versus maximum cores
%    utilized of s-parallel application with a sequential step followed
%    by a parallel section. \systemname is shown to be more efficient as cores
%    are scaled up due to wastage from the non-\systemname sequential portion of the application.}
%  \label{fig:wastage}
%\end{figure*}

%\if0
%\begin{figure}
%  \centering
%  \includegraphics[width=0.5\textwidth]{figs/synthetic_ehpcoverhead.pdf}
%  \caption{\small Percentile runtime increase of
%  Synthetic workflow  run with EHPC (including a scale down operation inside)
% compared running it staticaly wrapped with DTMCP.
%  Overall workflow runtime overhead induced by \systemname operations only.
%  Runn in Gordon. Workflow is Synthetic composed by a parallel and
%  sequential stage. Workflow was run with \systemname
%  and DMTCP, percentages. Figure express the increase in runtime when running
%  Synthetic with EHPC (and a scale operation inside) over a DTMCP static run.
%  With \systemname a scale down operation was performed between stages, runtime includes
%  checkpointing and queue wait time. \fix{To Be Removed- New data in Table~\ref{table:ehpc_overhead}} }
%%  \small \fix{This graph is actually showing nothing- 1. This is not for runtime elasticity, 2. The non-EHPC version ran under DMTCP, but without checkpointing; so had all the overheads of DMTCP monitoring and system call traps} Effect of runtime elasticity on synthetic workflow: The
%%    total core hour requirements versus maximum cores
%%    utilized of s-parallel application with a sequential step followed
%%    by a parallel section. \systemname is shown to be more efficient as cores
%%    are scaled up due to wastage from the non-\systemname sequential portion of the application.
%        \label{fig:synthetic_ehcp_overhead_synth_gordon}
%\end{figure}
%\fi

\subsection{Effect of Runtime Elasticity}
%\fix{Added section commented out that was on wrong graphs.}
%Fig \ref{fig:synthetic_corehrs_gordon} represents a prototypical 
%example of core hour efficiency that can be gained by incorporating E-HPC into
%an application with a varied topology of resources.  In the case of Fig
%\ref{fig:synthetic_corehrs_gordon} it can be seen that E-HPC maintains 
%core hour utilization within 22 percent of the single core execution. The
%Non E-HPC 128 core version shows a core hour utilization over five times as high as its
%16 core Non E-HPC execution. 

Figure \ref{fig:blast_execution} shows the use of \systemname
%at runtime on BLAST, a highly parallel workflow, where the 
for inducing elasticity in the middle of a workflow stage (runtime
elasticity), as it expands from 16 cores (one node on Gordon) into a
larger set.  Although \systemname is capable of scaling up in the
middle of a workflow stage, the results in the figure show that the
total workflow runtime is affected when using \systemname. The
sequential stages in BLAST are extremely short in comparison to the
longer parallel stage. The runtime for 16 cores with and without
\systemname are similar because in both cases, all the stages use 16
cores (one node on Gordon). However, as we scale up to 32 cores,
\systemname takes $\approx 6\%$ more time. When using \systemname the
first sequence stage, 1-BLAST is executed on one node %\fix{16 cores (= 1  node)},
and the elasticity is induced after 60 seconds. During this
time, some of the tasks in the second stage, 2-BLAST-Scaled, which is
a parallel stage, have already started executing. The higher degree of
parallelism during the initial parallel stage and the
checkpoint/restart overhead of DMTCP, the overall runtime performance
deteriorates with \systemname.  The pattern continues for larger
cores, and for up to 256 cores (with \systemname taking $\approx 20\%$
more time than without \systemname), \systemname runtime elasticity
performs poorly compared to when executed without \systemname.  This
is a significant result, because it shows that the time when
elasticity is induced is also critical to certain workflows, and may
result in performance degradation if the required resources are not
allocated at the right time.

%\if0
%Figure~\ref{fig:synthetic_corehrs_gordon} shows the core-hours used with and
%without \systemname, when using runtime elasticity. As the sequential stage,
%1-BLAST is extremely small, and the parallel stage, 2-BLAST-Scaled is extremely
%large, there are more resources allocated to the parallel stage for a longer
%duration when the workflow is executed with \systemname. Since, there is an
%increased runtime of $\approx 6\% - 20\%$ with \systemname for the parallel stage,
%the overall core-hours usage also increases significantly when scaling from 32 to
%256 cores.
%\fi
%The effect of performing this expansion is variable. In the transition to 32 cores, improvement in turnaround time is observed versus
%maintaining the inlitial resources.  This improvement is at a cost of 28 percent 
%in efficiency of core hour utilization and improvement of 40 percent in time.
%Looking at the larger core resource request, queue times increment dramatically, showing slower 
%turnaround in addition to wasted resources as demonstrated by the cost 
%of higher resource requests in Fig \ref{fig:blast_charged}.  
%The drawbacks of large queue times utilizing E-HPC is why Fast Mode 
%of E-HPC was devised.
%\fix{Add runtime elasticity (Gordon) results here.}



\vspace{-0.3cm}
\subsection{\systemname Overheads}
\label{section:ehpc_overheads}
%\fix{WF:Still needs to be consolidated, but this summarizes the graphs a little better if not more succintly}
%\fix{LR:Mostly Good text but hanging a bit. Let us consolidate and rearrange the
%  Gordon and Cori parts.}
%\fix{Devarshi: should possibly go at the beginning of results}

In this section, we evaluate the different overheads in \systemname.
Table~\ref{table:ehpc_overhead} shows the runtime overhead of various
workflows, with and without the queue wait times, when running with
\systemname on both Cori (C) and Gordon (G).  As \systemname resubmits
a job while scaling up, it incurs an additional queue wait time in
addition to the checkpoint and restart overheads of DMTCP. As can be
seen from the table, the overheads including the queue wait time are
significantly higher than excluding the wait time (for e.g., runtimes
are $86.3 \%$ longer with queue wait time vs $10.3\%$ longer without
the queue wait time for Montage on Cori). This is because the queue
wait time dominates the overheads in these cases and is a
system-dependent variable on which neither \systemname, nor DMTCP have
any control. On the other hand, the overheads without the queue wait
time only include DMTCP checkpoint and restart times, which has a
maximum of $\approx 36 \%$ overhead (for BLAST on 256 cores on
Gordon). BLAST is a memory-intensive application, with a large memory
footprint that generates large checkpoint images. The overheads are
smaller on Cori ($\approx 13 \%$) than on Gordon, because of the large
I/O bandwidth of the Lustre file system (700 GB/s), as compared to the
peak I/O bandwidth on Gordon (100 GB/s). % \fix{ADD THE LINK}.
For all other workflows, the runtime overhead varies between $0.2 \% - 11 \%$,
when there are no queue wait times. Hence, with current advancements
in storage system (e.g., burst buffers) and checkpoint restart
systems, \systemname overheads can be minimized.
% by speeding up
%the checkpoint and restart process.
%In all cases, the total runtime overhead in \systemname is less than $1\%$. The table measures
%purely the overheads in \systemname for managing the workflow
%migration across the resource slots. Hence, the runtime
%overhead consists of checkpoint and restart overheads in DMTCP,
%inter-stage queue wait times etc.
\begin{table}[!t]
\centering
\begin{tabularx}{0.48\textwidth}{cc|XXXX}
\hline
& & \multicolumn{4}{c}{\% Overhead, Wait (Without wait)} \\ \hline
Workflow & Sys. & 32 & 64 & 128 & 256\\
%\%Over.(nowait) & Sys. & n=32 & n=64 & n=128 & n=256\\
\hline
Montage & C & $N/A$ & $86.1(10.3)$ &$32.8(10.3)$ & $42.3(11.1)$\\
%Montage & Cori (no wait) & $N/A$ & $10.3$ &$10.3$ & $11.1$\\
BLAST & C & $N/A$ & $5.7(3.9)$ & $10.5(7.6)$ & $18.3(13.6)$\\
%BLAST & Cori (no wait) & $N/A$ & $3.9$ & $7.6$ & $13.6$\\
Synth& C & $N/A$ & $13.8(0.29)$ & $21.4(0.36)$ & $3.5(0.37)$\\
%Synth Expand & Cori (no wait) & $N/A$ & $0.29$ & $0.36$ & $0.37$\\
BLAST & G & $13.11(9.3)$ & $448(10.9)$ & $2085(13.4)$ & $5210(36.3)$\\
%BLAST & Gordon (no wait) & $9.3$ & $10.9$ & $13.4$ & $36.3$\\
Synth & G & $4.5(0.8)$ & $4.7(1.8)$ & $5(2.0)$ & $N/A$\\
%Synth Contract & Gordon (no wait) & $0.8$ & $1.8$ & $2.0$ & N/A\\
\hline
\end{tabularx}
\caption{
\small \systemname overheads including (left)
and excluding system dependent wait times (brackets).
 \systemname controlled overheads vary between $0.2 \% - 36\%$.
BLAST supports higher overheads due to its larger memory footprint and
hence, larger checkpoints.
%Gonzalo edit
%\small \systemname overheads. Overheads including
%the queue wait time are higher (values on the left) due to
%the skewness of system workloads, which can not be controlled
%by \systemname. Overheads excluding the queue wait time (values
%on the right), are solely due to the checkpoint and restart overhead
%in DMTCP. BLAST has high overheads due to a large memory footprint and
%hence, large checkpoints. \systemname overheads vary between $0.2 \% - 36\%$,
%when excluding queue wait times, depending upon the workflow characteristics
%and the file system performance. 
%%EHPC-over-DMTCP runtime overhead 
%%for Montage, BLAST, Synthetic in Cori and Gordon,
%%run with resource caps of 32, 64, 128, and 256 cores:
%%%EHCP-only overhead in workflow's runtime.
%%Each value represents the runtime increase in percentile
%%because of using EHPC (checkpointing/restart) with
%%and without wait times.
%%It does not include DMTCP monitoring overhead that might
%%slow down code execution.
%%
%% Lines in this figure coorpond to data from
%% Figs, 7,8,9,10,11 (defunct soon)
%%
%%
}
\label{table:ehpc_overhead}
\vspace{-0.6cm}
\end{table}

Figure~\ref{fig:time_dmtcp}) shows the overheads in \systemname
due to the checkpoint and restart phases.  Both checkpoint and restart
overheads are proportional to the number of workflow tasks in
execution, and the overheads increase linearly up to 150
tasks/processes. The overhead is due to the added communication between the
workflow tasks and the DMTCP coordinator, and the I/O overhead of
writing the checkpoint image to disk.

%This is due to the added communication between
%the workflow tasks and the DMTCP coordinator. Hence, the time overhead
%is proportional to the number of workflow tasks in execution, and
%keeps increasing with increasing number of tasks. However,
%DMTCP's efficient checkpoint/restart mechanism incurs relatively
%small time overheads with increasing multi-node tasks/processes. 

Figure~\ref{fig:size_dmtcp}) shows that the storage space overhead
also increases linearly with the increasing number of tasks. 
The total amount of memory and compute requirements increase
with increasing tasks, thereby increasing the total checkpoint size.
An important observation from Figure~\ref{fig:size_dmtcp} is that the
checkpoint size may become so large that it can result in I/O performance
bottlenecks that can significantly affect the overall \systemname performance. 
%\fix{DG: the sentence below may go in future work.}
DMTCP provides optimizations for writing checkpoint images to memory,
and also provides compressed checkpointing to minimize the memory and storage
footprint of checkpoints and restarts. These optimizations can be used to
minimize the overheads in \systemname. 

%We plan to address these optimizations
%in future versions of \systemname.

%This provides further support for utilizing expand and contract at model low thread count locations in a process. This applies to the Blastall process where checkpoint and restart often occurs during parallel stages of execution.  \fix{WF: (Hesitant to include the obvious/following) Combined, a user would sacrifice compute efficiency, potentially throttle I/O performance of the system, in order to gain access to a greater number of cores and possibliy reduce turnaround time. }

%The minimum required time for a checkpoint and restart, disregarding
%the queue wait time, is under ten seconds when checkpointing a single
%task of the workflow. It was also shown that the type of
%application can have predictable but varying overhead as shown in Fig
%\ref{fig:size_dmtcp} and Fig \ref{fig:time_dmtcp}. Due to DMTCPs
%efficient methods, as a job grows across nodes, the rate of change
%of overhead time and size decreases.
%
%Blastall in Fig \ref{fig:blast_execution} shows that when initiating a highly scalable process currently in a parallel section, improved turnaround time can be obtained by scaling to 128 cores as opposed to having started with 64 cores in the first place. Improved turnaround time incurrs a resource efficiency cost where Fig \ref{fig:blast_charged} displays the resource cost of performing an elastic expansion from 16 cores to the relative peak value. Looking at a more realistic scenario Fig \ref{fig:elastic_demand} shows proof of concept that ehpc expanding from a high core count to a higher core count improves turnaround over staying at the lower core count. 
%
%An important observation from Fig \ref{fig:size_dmtcp} is that the footprint of a checkpoint on disk grows to a size large enough where I/O congestion on the cluster can have significant detrimental impacts to the overall performance of EHPC. This provides further support for utilizing expand and contract at model low thread count locations in a process. This applies to the Blastall process where checkpoint and restart often occurs during parallel stages of execution.  \fix{WF: (Hesitant to include the obvious/following) Combined, a user would sacrifice compute efficiency, potentially throttle I/O performance of the system, in order to gain access to a greater number of cores and possibliy reduce turnaround time. }
%\begin{figure}
%\centering
%\includegraphics[width=3.18in]{figs/time_dmtcp.png}
%\caption{Time required for checkpoint and restart vs. the number of threads being tracked by EHPC.}
%\label{fig:time_dmtcp}
%\end{figure}
%\begin{figure}
%\centering
%\includegraphics[width=3.18in]{figs/size_dmtcp.png}
%\caption{Total storage required on filesystem for a checkpoint versus the number of threads being tracked by EHPC.}
%\label{fig:size_dmtcp}
%\end{figure}

%\if0
%\fix{Why is this labelled summary} 
%
%\noindent\textbf{Summary} \fix{This part needs to be fixed. I will
%  discuss it with Devarshi, since they are related to William's
%  experiments in Gordon}
%
%\fix{Overall this text reads well... I think questions is where does it fit.} 
%
%\fix{Figure X shows ...} \fix{Also need to be explicit this was on Gordon.}
%The overall efficiency of resource utilization for each application is
%shown to allow for significant improvement under \systemname as shown
%in Fig \ref{fig:wastage}.  When tracking core hours, \systemname was
%shown to maintain nearly even efficiency of resource utilization, even
%when the total number of cores required was increased eight fold.  The
%increased wall time, or slowdown, was shown to be within 0.2 and 1.7
%percent of the total wall time, where losses through \systemname were
%primarily due to the checkpoint and restart phase, when excluding
%queue times.
%
%The total observed efficiencty in Fig \ref{fig:wastage} shows that a
%single node without \systemname is the baseline for optimal resource
%utilization of the synthetic benchmarks.  As resources grow, it is
%shown that without \systemname resource utilization increases by a
%factor of five, while incorporating \systemname with matched resource
%availability shows an increase of resource utilization of 20
%percent. As cores are increased and jobs become more dynamic in their
%resource uses, \systemname proves to become increasingly effective at
%maintaining resource availability, while decreasing resource wastage.
%\fi

\vspace{-0.2cm}
\subsection{Summary}
In this section, we summarize the experimental results.
%\fix{reverify this once because of changes to table}
%\begin{tightItemize}
%\item 
The workflow runtimes are $\approx 6\% - 20\%$ time longer in \systemname as compared
to running the workflow without \systemname. The runtime results for
the workflows show that the performance of workflows with \systemname
is affected due to the checkpoint-restart overhead, queue wait time and the underlying
application characteristics (Figure~\ref{fig:montage_turnaround-new}, Figure~\ref{fig:montage_core-hour-new}). 

%\item 
\systemname improves the core-hours used for running the workflows by up to
$76 \%$. The core-hour results show that with increasing parallelism,
and longer sequential stages, \systemname utilizes resources more efficiently
than its counterpart by allocating only as many resources as needed for
a stage in the workflow (Figure~\ref{fig:blast_turnaround-new}, Figure~\ref{fig:blast_core-hour-new}).
%\item 

The runtime overheads in \systemname vary between $0.2 \% - 36 \%$, when excluding the
highly variable queue wait times. Further evaluation shows that the overheads are solely
due to the underlying file system, and DMTCP (checkpoint/restart library) (Table~\ref{table:ehpc_overhead}).
%\end{tightItemize}

%\if0
%
%\item \systemname is shown to maintain nearly even efficiency of resource
%utilization, even when the total number of cores required was increased eight
%fold. (\fix{Figure~\ref{}})
%\item The increased wall time, or slowdown, is shown to be within 0.2 and 1.7
%percent of the total wall time, where losses through \systemname are
%primarily due to the checkpoint and restart phase, when excluding
%queue wait times. (\fix{Figure~\ref{}})
%\item The total observed efficiency between non-\systemname and \systemname
%shows non-\systemname workflow execution on a single node is the baseline
%for optimal resource utilization of the synthetic benchmarks. (\fix{Figure~\ref{}})
%%\item The results also show that without \systemname, resource utilization
%%increases by a factor of five, while incorporating \systemname with matched resource
%%availability shows an increase of resource utilization of 20
%%percent. 
%\item As cores are increased and jobs become more dynamic in their
%resource uses, \systemname proves to become increasingly effective at
%maintaining resource availability, while decreasing resource wastage.
%(\fix{Figure~\ref{}})
%
%\fi


