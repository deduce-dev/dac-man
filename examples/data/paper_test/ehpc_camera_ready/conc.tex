In this paper, we present the design, implementation and evaluation of
\systemname, a flexible elastic framework that provides increased efficiency
of resource utilization, failure recovery, elasticity, and faster
execution times. The overheads of \systemname vary based on the
characteristics of the application (e.g., amount of I/O), resource
configuration and run-time characteristics (e.g., queue wait
time). The overheads are from checkpoint and restart, while providing
significant benefits in dynamic elastic resource
management. \systemname is designed to work independently as well as
with existing software ecosystems. \systemname provides an effective
library for the fine tuned control of resources in HPC environment,
where before now, real time control was difficult, if not
impossible. \systemname is the foundational tool needed to address the
resource management needs of next-generation real-time and streaming
workflows.


% at optimal moments of execution and produce predictable overhead
% when being utilized in scenarios with a greater number of
% checkpointed processes.  DMTCP and EHPC are both extremely flexible
% in the frameworks and schedulers they can handle effectively
% lowering the barrier of introducing elasticity within traditional
% HPC environments and workflows. EHPC also has the
%benefit of giving new insights into the topology of workflows being
%run, including execution times of discreet sections of code and
%process congestion throughout the workflow.  


%Future work with EHPC will include testing on effects of EHPC on
%overall queue times experienced by submitting smaller jobs, cost
%saving possible on cloud based batch schedulers, and incorporation
%into existing scientific applications.
