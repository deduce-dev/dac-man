Today, scientific workflows with experiment data are increasingly processed on High
Performance Computing (HPC) systems. Next-generation scientific
workflows have to support streaming data and real-time constraints with
varying resource needs.  Today, HPC platforms are primarily designed
to support monolithic MPI applications and provide a static resource
allocation model i.e., a resource allocation is fixed for the duration
of the entire job. The static resource allocation model 
presents challenges to scientific workflows where every stage of the
workflow may have different resource needs. Current allocation
methods attempt to either split the workflow and incur queue wait
times for each stage, or request one big allocation resulting in
wastage of resources.

\begin{figure}[htbp]
\centering
\includegraphics[width=0.45\textwidth]{figs/NonEHPCvsEHPCTurnAroundTime.png}
\caption{\small Comparison of traditional and \systemname managed resource allocation
 for scientific workflows in HPC: a) shows the static allocation of 
 resources for the entire duration of the workflow execution.
 b) shows the dynamic allocation where resources are requested as per the needs of a 
 particular stage.}
% Gonzalo edit
% \caption{\small Traditional vs \systemname managed resource allocation
% for scientific workflows in an HPC environment. a) statically allocates
% a fixed set of resources for the entire duration of the workflow execution.
% b) dynamically allocates only as many resources as required by a
% particular stage of the workflow.}
\label{fig:ehpc_overview}
\vspace{-0.5cm}
\end{figure}
 
Current methods result in loss of efficiency or utilization, and the
problems will only be exacerbated with next-generation dynamic
workflows. We need a dynamic resource management model that considers
resources to be elastic that can grow and shrink based on
requirements of a scientific workflow. Resource elasticity has
been extensively studied in the context of
clouds~\cite{sharma2011cost, galante2012survey, paraiso2016socloud}.
However, unlike HPC environments, cloud resources are not managed
through batch schedulers. Elastic resource management in HPC environments
has also been explored for specific applications~\cite{hpc_elastic_example1},
but general methods are still not available.

In this paper, we present \textbf{Elastic-HPC (\systemname)}, an
elastic framework for managing resources for scientific workflows in
an HPC environment. It provides dynamic, adaptable resource
management and supports workflow execution. E-HPC is capable of
growing and shrinking the allocated resources for a workflow during
execution. It considers the resource slot to be an elastic window that
maps to different physical resources based on availability.
\systemname uses checkpoint-restart mechanism to save and relaunch
workflows on different resources when needed. Users can either submit a
workflow description to \systemname, or instrument existing workflows to
use \systemname to manage elastic resources. Thus, \systemname can fit
into current software ecosystems available in scientific collaboration
or at HPC facilities. In this paper, we make two contributions.

\begin{itemize}  
\item We design and implement an elastic framework to manage resources
for scientific workflows in HPC environments.
\item We evaluate the performance of the \systemname framework across two
HPC systems, and several synthetic and real scientific workflows to
understand the performance of workflows and overheads in \systemname. 
\end{itemize}  

The rest of the paper is organized as follows. In
section~\ref{sec:background}, we describe background on workflows
and current methods of resource management. We describe the
design and implementation of \systemname framework in
Section~\ref{sec:arch} and results of our evaluation in
Section~\ref{sec:results}. We present related work in
Section~\ref{sec:related} and conclusions in
Section~\ref{sec:conc}.



%Scientific workflows are
%often executed by chaining multiple jobs together and allocating the
%required resources prior to their execution on HPC
%resources~\cite{}. The rigidity of managing resources for scientific
%workflows is the major hindrance to incorporate elasticity in HPC
%environments.



%There is a need for an elastic
%framework to support the dynamic and real-time needs of scientific workflows
%running on HPC resources.

%On existing HPC systems, workflow and resource management techniques are
%disconnected and result in static allocation of resources for the entire
%duration of the workflow. Current approaches for running workflows in an HPC
%environment have several major issues. First, they have high latency
%and turnaround time due to resource limitations of the traditional
%batch queue systems. Second, they often result in an inefficient and
%underutilized system due to over allocation of resources during
%workflow execution. Third, users often incorrectly estimate the execution
%time of their workflows, resulting in job failures and manual restarts.
%Finally, existing tools do not allow workflows to dynamically scale and
%hence, do not adapt to the varying data and resource requirements
%of long-running scientific workflows.
%\fix{where do the triggers come from? what happens when inside the app?
%what happens when outside the app?}



